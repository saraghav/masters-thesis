\chapter{Conclusions and Future Directions}
\label{chap:conclusion}

\section{Conclusion}

In this thesis, I illustrate applications of ideas from Machine Learning research in three different areas: edge computing systems, wireless systems research and mobile systems research.
 In all three avenues, there is tremendous scope to define new problems that can be tackled by Machine Learning drive by the volume of data-driven analyses already occurring in practice.


In case of water hardness detection, the machine learning algorithms are already deployed in real systems by virtue of the Madison based startup Emonix Inc \footnote{The company can be found online at \url{https://www.emonix.io}.}.


In case of wireless spectrum estimation, this is an actively research unsolved problem with tremendous promise.
 In my work, I describe the trappings of an initial solution using matrix completion.
 However, there are several ideas to improve the currently achieved performance, as described in Section \ref{sec:futureWork}.


In case of privacy preserving keyphrase recognition, to the best of my knowledge, this is the first line of work that examines keyphrase recognition using lightweight audio obfuscation techniques.
 I hope to carry my initial results here in building a privacy-preserving keyphrase recognition service that can be deployed in mobile systems with minimal effort.

\section{Directions for Future Work}
\label{sec:futureWork}

\subsection{Water Hardness Detection}

\subsubsection{Characterizing transient water usage patterns}

One shortcoming of the existing water flow forecasting approach is that it is unable to predict the instantaneous flow because of the high-order dynamics associated with the signal.
 Recent research suggests moderately complex LSTMs (Long-Short Term Memory Networks) are able to learn compact representations of high-order dynamics.
 This may be an interesting avenue to explore in future work.

\subsubsection{Collective water quality characterization}

With real systems being deployed in improving water softener efficiency, it could be valuable to provide recommendations to the local water authority to improve water quality and resource utility.
 Using data collected from neighboring installations of the Emonix system, it would be possible to build a database of water quality and usage history with geographic tags.
 This can be used by the local water authority for understanding drawbacks, if any, in the water distribution infrastructure as well as notify consumers of excessive water usage patterns.


\subsection{Wireless Spectrum Estimation}

\subsubsection{Incorporating physical propagation models}

One of the drawbacks of the estimation method proposed in this thesis is the 5dBm estimation error.
 It would be interesting to consider using wireless propagation models along with geographically diverse data (as opposed to using data only from a single location as described in Chapter \ref{chap:spectrum}).
 This would be equivalent to a regularization technique that leverages physical models and allows for more data to factor into the estimation.


\subsection{Privacy-sensitive Acoustic Perception}

\subsubsection{Theoretical Analysis based on Differential Privacy}

The results from the dbHound system presented in Chapter \ref{chap:sound} discuss empirical evaluation of keyphrase recognition from obfuscated audio.
 However, for strong guarantees on privacy and security, it is important to analyze the obfuscation techniques using the lens of differential privacy.
 Such an analysis would help to quantify the degree of privacy and security improvements in the context of the threat models described in Section \ref{sec:threat_model}.
