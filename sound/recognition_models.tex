
\section{Keyphrase Recognition from Obfuscated Audio}
\label{sec:recognition_models}

I propose a deep neural network architecture (Figure \ref{fig:convnet_architecture}) to achieve this keyphrase recognition task in real time.
 Research work from Google in the past \cite{smallfootprintkeywordspotting} \cite{cnnkeywordspotting} \cite{chen2015locallyconnected} \cite{chen2014low} have shown that deep convolutional neural networks are suitable for online keyphrase recognition in mobile devices.
 The focus of my work is to survey the trade-off between privacy and keyphrase recognition performance for different audio obfuscation techniques.
 Based on the work from Google, I add the remark that efficient implementations of such architectures have been realized for mobile devices and it will be possible to translate the architecture I propose in a similar fashion, to mobile devices to realize the recommendations in Section \ref{sec:proposed_changes_to_mobile}.

The different layers of the deep convolutional neural network are described in detail below.

\subsection{Input Layer}

This contains all the samples from the obfuscated audio for a 2.048 second interval, with the audio being sampled at 16 kHz with 16-bit resolution.
 The interval of 2.048 seconds was chosen so that the corresponding number of raw audio samples is a power of 2 (32768 raw audio samples).
 This choice is purely one of convenience for signal processing.


Based on the obfuscation technique applied, the total number of samples provided in the input layer may be different.
 Some examples for this are shown in Table \ref{tab:input_layer_samples}.
 However, all of these variants still correspond to a time interval of 2.048 seconds, assisting in the consistency of the deep convolutional neural network across different obfuscation techniques.


For the sake of clarity when working with convolutional layers, the shape of the input layer can be re-written in the following manner.

\begin{gather}
\text{Shape Template} = (\text{height} \times \text{width} \times \text{channels}) \\
\text{Shape of Input Layer} = (N \times 1 \times 1)
\end{gather}

\begin{table}[!th]
\centering
\begin{tabularx}{\textwidth}{l | l}
\textbf{Obfuscation Technique} & \textbf{Input Layer Size} \\
\hline
No Obfuscation (Raw Audio) & 32768 \\
Decimation to 4 kHz & 8192 \\
Decimation to 1 kHz with 8-Hamming Reduction & 1024 \\
Decimation to 1 kHz & 2048 \\
Decimation to 1 kHz with 2-Hamming Reduction & 1024 \\
Bit Depth to 1 bit & 32768 \\
Bit Depth to 1 bit with 8-Hamming Reduction & 4096 \\
\end{tabularx}
\caption{Example list of size of input layer in the Deep Convolutional Neural Network for various obfuscation techniques}
\label{tab:input_layer_samples}
\end{table}

\subsection{Convolutional Layers}

In this case, I propose using 3 cascaded convolutional layers (interspersed with 3 max-pooling layers) to learn audio features for the classification task.
 Since the "width" (from the shape template above) is always 1, these layers can also be likened to a 1-D convolutional operation.
 Each convolutional layer is completely defined by the parameters described below. \footnote{Chris Olah's explanation in \cite{cnn_modular} is a great resource to understand the structure of ConvNets.}

\begin{itemize}
\item \textbf{Filter Size $(S)$:} The size of the convolutional filter defines the number of samples to work with to generate a feature.
 This also decides the size of the weight matrix for a single convolutional filter.
 For example, if only one filter of size $(20 \times 1 \times 1)$ is present, the filter works with 20 audio samples and 1 channel
 In other words, 20 weight variables are learnt per filter.
 The filter width is always 1, since this is a 1-D signal, and typically the filter works with all the channels of the input.
 Consequently, the filter size $S$ can be defined by a single number, 20, in this case.

\item \textbf{Filter Strides $(P)$:} The term "convolution" applies here because of the striding operation.
 Once the size of the convolutional filter is decided, the filter walks over sub-sections of the input space to compute a feature for different parts of the input.
 This can be likened to computing the Fourier Transform of different portions of a signal to understand localized frequency distributions --- the convolutional filter learns to find the best localized features.
 Clearly, given the high sample frequency of audio signals, striding the filter sample by sample would result in excessive computational time.
 Moreover, striding sample by sample is not likely to provide accuracy benefits either --- owing to the high sample frequency, the information (or phoneme) content is dispersed.
 The Filter Strides $P$ is the number of samples the filter "walks" over before computing the next feature.


Decreasing this parameter results in the filter striding in shorter steps (more instances of the same filter), increasing computation time while learning and evaluating.
 The size of the model (in terms of memory), however, remains unchanged.

\item \textbf{Number of Filters $(K)$:} This is used to set the number of filters (different features) to learn from the input.
 This can be likened to computing different types of transforms (e.g. fourier, haar, daubechies, cepstral etc.) over the same sub-sections of the input data.
 The intuition behind this is that different filters are likely to capture the essential features of the signal when used together.

\item \textbf{Activation Function:} This is the activation function used to generate the output at each node.
 Typical choices are ReLU, Sigmoid, tanh etc.

\end{itemize}

The size of the output of the 1-D convolutional filter is described below.
 The structure of my proposed deep convolutional neural network is described in \ref{tab:conv_layer_list}.

\begin{gather}
\text{Shape Template} = (\text{height} \times \text{width} \times \text{channels}) \\
\text{Shape of Input Layer} = (N \times 1 \times C) \\
\text{Shape of Output Layer} = \left(\frac{N-S+1}{P}, 1, K \right)
\end{gather}

The max-pooling layers are equivalent to a subsampling operation.
 They are introduced to maintain local translational invariance in the model.
 Consequently, a max-pooling operation of 4 decreases the number of input samples by a factor of $\left(\frac{1}{4}\right)$ in the output.
 The number 4 here is referred to as the "kernel size" of the max-pooling layer.
 It is standard practice for a max-pooling layer to follow a convolutional layer.
 \footnote{The Convolutional Layer together with its Max-Pooling Layer will be referred to as Conv-Max layer in the rest of this thesis.}
 All the max-pooling layers used in the architecture in Figure \ref{fig:convnet_architecture} have kernel size 2.

\begin{table}[!th]
\centering
\def\arraystretch{2}\tabcolsep=10pt

\begin{subtable}{\textwidth}
\centering
\begin{tabular}{l | l | l}
\textbf{Parameter} & \textbf{Convolutional Layer: 1} & \textbf{Convolutional Layer: 2} \\
\hline
Filter Size & $S = \frac{N}{128} \sim 16 ms$ & $S = 2 \sim 32 ms$ \\
Filter Strides & $P = \frac{S}{2} = \frac{N}{256}$ & $P = \frac{S}{2} = 1$ \\
Number of Filters & 1024 & 512 \\
Activation Function & ReLU & ReLU \\
Regularization & $L_2$ & $L_2$ \\
\end{tabular}
\end{subtable}

\vspace{3cm}

\begin{subtable}{\textwidth}
\centering
\begin{tabular}{l | l}
\textbf{Parameter} & \textbf{Convolutional Layer: 3} \\
\hline
Filter Size & $S = 8 \sim 256 ms$ \\
Filter Strides & $P = \frac{S}{2} = 4$ \\
Number of Filters & 128 \\
Activation Function & ReLU \\
Regularization & $L_2$ \\
\end{tabular}
\end{subtable}

\caption{Description of filter size for the Deep Convolutional Neural Network Architecture in Figure \ref{fig:convnet_architecture}. The input layer has $N$ samples.}
\label{tab:conv_layer_list}
\end{table}

\subsection{Densely Connected Layers}

The cascaded convolutional layers are used to learn the "best" features for the classification task at hand.
 The densely connected layers are intended to learn the actual classification task from the "best" features.
 If the output of the cascaded convolutional layers are considered as "features", then the rest of the network is a two-layer perceptron.
 In other words, the rest of the network can be treated as a neural network with one hidden layer and one output layer.

The hidden layer has ReLU activation with dropout regularization (the dropout probability is chosen as 0.8).
 The output layer has a softmax activation function to compute class probabilities.


When the neural network is fine-tuned online by the dbHound system, only the weights for the densely connected layers (two-layer perceptron) are updated.
 The reason only the two-layer perceptron is fine-tuned is to prevent overfitting on new data --- using this approach, the feature computation is "frozen" and the classification boundary is \textbf{slightly} nudged to accommodate new examples.
 This "incremental" learning is also done with a very small learning rate with the idea that the classifier is already well learnt and only needs to be slightly adapted to deal with new data.

\begin{figure}[!t]
\centering
\tikzset{%
    cascaded/.style={%
        general shadow={%
            shadow scale=1,
            shadow xshift=-1ex,
            shadow yshift=1ex,
            draw,
            thick,
            fill=white
        },
        general shadow={%
            shadow scale=1,
            shadow xshift=-.5ex,
            shadow yshift=.5ex,
            draw,
            thick,
            fill=white
        },
        fill=white,
        draw,
        thick,
        minimum width=1.5cm,
        minimum height=2cm
    }
}

\tikzstyle{vecArrowShort}=[thick,
                      shorten >= 5.5pt,
                      postaction={draw,line width=2pt, black, shorten >= 4.5pt}
]
\tikzstyle{vecArrow}=[thick,
                      postaction={draw,line width=2pt, black}
]

\tikzstyle{maxpool} = [rectangle, draw, fill=yellow!20, text centered, rounded corners]
\tikzstyle{conv} = [rectangle, draw, fill=blue!20, text centered, rounded corners, minimum height=4em, minimum width=20em]
\tikzstyle{shortline} = [draw, -latex', fill=black, vecArrowShort]
\tikzstyle{line} = [draw, -latex', fill=black, vecArrow]
\tikzstyle{dense} = [draw, ellipse,fill=red!20, minimum height=2em]
\tikzstyle{input} = [rectangle, draw, fill=green!20, text centered, sharp corners, minimum height=2em, minimum width=15em, align=center]
\tikzstyle{output} = [rectangle, draw, fill=green!20, text centered, sharp corners, minimum height=2em, minimum width=5em, align=center]

\begin{tikzpicture}[node distance = 2cm, auto]

% nodes
\node [input] (input) { $N \text{Audio Samples}$ };
\node [cascaded, conv, below of=input] (conv1) { $\text{1-D Convolutional Layer: 1}$ };
\node [maxpool, below of=conv1] (maxpool1) { Max-Pool Layer: 1 };
\node [cascaded, conv, below of=maxpool1] (conv2) { $\text{1-D Convolutional Layer: 2}$ };
\node [maxpool, below of=conv2] (maxpool2) { Max-Pool Layer: 2 };
\node [cascaded, conv, below of=maxpool2] (conv3) { $\text{1-D Convolutional Layer: 3}$ };
\node [maxpool, below of=conv3] (maxpool3) { Max-Pool Layer: 3 };
\node [dense, below of=maxpool3] (dense1) { $\text{Fully Connected Layer: 1}$ };
\node [dense, below of=dense1] (dense2) { $\text{Fully Connected Layer: 2}$ };
\node [output, below of=dense2] (output) { $(2 \times 1)$ softmax output };

%arrows
\path [shortline] (input) -- (conv1);
\path [line] (conv1) -- (maxpool1);
\path [shortline] (maxpool1) -- (conv2);
\path [line] (conv2) -- (maxpool2);
\path [shortline] (maxpool2) -- (conv3);
\path [line] (conv3) -- (maxpool3);
\path [line] (maxpool3) -- (dense1);
\path [line] (dense1) -- (dense2);
\path [line] (dense2) -- (output);

\end{tikzpicture}
\caption{Architecture of the Deep Convolutional Neural Network for Keyphrase Recognition}
\label{fig:convnet_architecture}
\end{figure}

