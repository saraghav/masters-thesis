
\section{Introduction}

Existing spectrum databases \cite{GoogleSpectrumDatabase} \cite{SpectrumBridge} \cite{MicrosoftWhitespaces} in the USA are populated based on RF propagation models.
 Consequently, the databases are likely to have inaccurate estimates of spectral power and occupancy.
 There have been wealth of spectrum measurement efforts aimed at recording ground truth data of spectrum usage to better quantify utilization and occupancy and improve the propagation models.
 These efforts have seen a lot of interest in the context of cognitive radio applications and dynamic spectrum management which are approaches to maximize spectrum utilization \cite{SpectrumForCogRadio1} \cite{ChicagoSpectrum} \cite{ChicagoSpectrum2} \cite{SpectrumDynamics}.


An important concern in deploying devices that support dynamic spectrum access is the identification of areas where spectrum utilization experiences a lot of temporal variations.
 Applications supporting dynamic spectrum access in such areas can take advantage of temporal variations in spectrum utilization to improve their connectivity.
 For example, small cell deployments supporting dynamic spectrum access can take advantage of different under-utilized frequency bands during the course of a day and improve backhaul bandwidth.
 Clearly, deployments supporting dynamic spectrum access call for online spectrum measurement for feedback to minimize the chances of interfering with other applications.

The first step to realize such deployments is to identify areas that see significant temporal variations and pick the best candidates from this information.
 I propose a method to study temporal variations in spectrum utilization by uncovering missing samples from opportunistic spectrum measurements.
 I show that opportunistic spectrum measurements are sparse by nature and propose three sources of sparsity associated with these measurements.
 Prior work by Zhang and Banerjee \cite{VScope} \cite{VScope2} discusses opportunistic spectrum measurements as a way to measure 'anchor points' to improve the information quality in spectrum databases.
 They describe a system 'VScope' which uses a spectrum analyzer placed in a public transit vehicle, to collect spectrum measurements across a city.
 I reuse the data gathered from VScope \footnote{I refer to the data gathered as part of the VScope \cite{VScope} project as "VScope data" for the rest of the thesis.} and examine its characteristics and propose solutions that uncover the missing samples with high accuracy.
 I also note that, with large amounts of spectrum data that has been gathered across the world \cite{petrin2004measurement} \cite{ChicagoSpectrum} \cite{ChicagoSpectrum2} \cite{SpectrumDynamics} \cite{GoogleSpectrumDatabase} \cite{MicrosoftWhitespaces} \cite{SpectrumIndia} \cite{SpectrumDataMining} I believe this approach would be useful to study temporal variations in spectrum utilization.

In this chapter, I discuss applying a technique called Matrix Completion \cite{CandesRecht} (from Collaborative Filtering literature) to uncover the missing samples.
 I justify the applicability of this technique and evaluate its performance on 6 months of sparse spectrum measurements collected in the City of Madison.
 I show that even with simplifying linearity assumptions on the spectrum measurement data, it is possible to uncover missing samples with an error less than 5 dBm in about 80\% of measurements.

By uncovering missing spectrum measurements using this approach, studying the dynamics of spectrum usage and trends in mobility is made possible.
 For wireless infrastructure development surveys it is useful to quantify the impact of new infrastructure based on location.
 I see value for this approach in this context as well, since uncovering missing samples allows for inferences based on a complete data set.
 I also see value in regulatory efforts for wireless spectrum since this approach could potentially uncover violations in spectrum usage.
 Additionally, this approach has low computational overhead and recent advances in Machine Learning allow for distributed implementations \cite{Hogwild} \cite{Cyclades}.

Sparsely sampled crowdsourced measurements reduce the burden of deploying comprehensive spectrum measurement devices.
 However, because of the missing samples, crowdsourced data or opportunistic measurements cannot be used "as is" for analyses.
 Our approach helps uncover actionable data from crowdsourced spectrum measurements \cite{WiSee} with a computationally light algorithm that can be implemented on mobile devices.
